\subsection{Thiết kế Protocol: }

Tại phần này, chúng tôi sẽ trình bày chi tiết giao thức mà Client và Server dùng để giao tiếp với nhau. 

\subsubsection{Thiết lập kết nối, cập nhật và ngắt kết nối}

\begin{itemize}
	\item \textbf{Kết nối: } Sau khi \textbf{Client} được khởi tạo và yêu cầu kết nối với \textbf{Server} đang \textit{Listening}, nếu yêu cầu được chấp nhận, \textbf{Server} sẽ phản hồi bằng cách gửi một \textit{message} có kiểu header là \textbf{SERVER\_ACCEPT}. \textbf{Client} nhận được phản hồi này và tiến hành xử lý các bước khởi tạo, hoàn tất việc thiết lập kết nối giữa hai bên.
	\item \textbf{Cập nhật: } Trong quá trình kết nối, cả hai đều liên tục trao đổi thông tin. Về phía \textbf{Server}, nó sẽ gửi các thông tin cập nhật đến màn hình Remote của \textbf{Client}, mỗi thông tin này đều được gửi dưới các \textit{message} có kiểu header là \textbf{SERVER\_UPDATE}. Về phía \textbf{Client}, có nhiều loại \textit{message} khác nhau được sử dụng để cập nhật thông tin cho \textbf{Server}. Chúng sẽ được mô tả chi tiết trong các phần tiếp theo của \textbf{Kiến trúc Protocol}.
	\item \textbf{Ngắt kết nối}Khi \textbf{Server} chủ động ngắt kết nối, nó sẽ gửi thông báo cho \textbf{Client} qua \textit{message} có kiểu header là \textbf{SERVER\_DISCONNECT}, nội dung rỗng. Tương tự, nếu \textbf{Client} chủ động ngắt kết nối, nó cũng sẽ gửi thông báo cho \textbf{Server} qua \textit{message} có kiểu header là \textbf{CLIENT\_DISCONNECT}.
\end{itemize}
	
\subsubsection{MetaData}

Giao tiếp này giúp người dùng bên \textbf{Client} và \textbf{Server} nhận được các thông tin cơ bản của nhau.

Khi \textbf{Client} được tạo, nó sẽ ngay lập tức gọi hàm để gửi các thông tin như \textit{địa chỉ IP, địa chỉ MAC},... cho \textbf{Server}, với \textit{message header} là \textbf{MetaData}.

Về phía \textbf{Server}, nó cũng sẽ gửi các thông tin cơ bản. Các thông tin này được gửi đi cùng với \textit{message} có  \textit{header} là \textbf{SERVER\_ACCEPT} đã được trình bày ở trên.

\subsubsection {Các sự kiện chuột, bàn phím}

Nội dung được trình bày sau đây là các loại \textit{message} chứa sự kiện chuột, bàn phím mà \textbf{Client} gửi tới \textbf{Server}.

Để thông tin được trình bày một cách rõ ràng và khái quát nhất, sau đây tôi sẽ liệt kê từng loại \textit{message header}, miêu tả thành phần nội dung được lưu trữ của \underline{tin nhắn mang header đó} và \underline{công việc của loại tin nhắn} đó.

\begin{itemize}
	\item \textbf{MouseClick}:  Dùng để gửi sự kiện chuột được nhấn. Chứa toạ độ tung, hoành nơi mà \textbf{Client} ấn chuột và loại nút chuột được nhấn \textit{(left, right, middle, aux1...)}.
	\item \textbf{MouseUnClick}: Dùng để gửi sự kiện chuột được nhả ra. Cũng chứa toạ độ và loại nút chuột được nhả.
	\item \textbf{DoubleClick}: Dùng để gửi sự kiện double-click được nhấn. Tương tự, chứa toạ độ và loại nút.
	\item \textbf{MouseMove}: Dùng để cập nhật vị trí con trỏ chuột. Loại tin nhắn này chỉ chứa toạ độ của chuột.
	\item \textbf{MouseWheel}: Dùng để cập nhật con lăn chuột. Chứa toạ độ và một giá trị chỉ hướng lăn của chuột.
	\item \textbf{KeyPress}: Dùng để gửi sự kiện phím được nhấn. Chỉ chứa một giá trị, đó là mã của phím.
	\item \textbf{KeyRelease}: Dùng để gửi sự kiện phím được nhả. Tương tự, chứa mã của phím.
\end{itemize}

\subsection{Capture màn hình}

Để phục vụ cho chức năng \textit{Chụp màn hình Server}, hai loại \textit{message header} \textbf{CaptureRequest} và \textbf{CaptureSend} được sử dụng.

Trong đó, khi \textbf{Client} thực thi sự kiện yêu cầu ảnh chụp màn hình từ \textbf{Server}, nó sẽ gửi một \textit{message} dưới \textit{header} \textbf{CaptureRequest}. Khi gửi lại hình cho đối phương, \textbf{Server} sẽ sử dụng \textit{header} \textbf{CaptureSend}  và nội dung của tin nhắn sẽ là mảng \textit{bit} của hình được chụp.



\subsection{Chụp màn hình}
Server sẽ gửi lệnh \textbf{LIVESCREEN} đến client, sau đó sẽ gửi lệnh tương ứng với thời gian cần chụp màn hình, thời gian mặc định là $0.5$ giây. Thời gian chụp không nên quá nhiều, vì giới hạn dung lượng tệp đính kèm trong email. Sau đó, server sẽ nhận ảnh gửi từ client, tạo thành một video.\\
Các hàm cho quá trình này như sau:
\begin{itemize}
\item Ở server (mail$\_$handler.py):
\begin{lstlisting}
def capture_screen(ip_address, time=0.5):
\end{lstlisting}
Chức năng: gửi yêu cầu chụp màn hình đến cho client, nhận hình ảnh từ client gửi lại, tạo video, gửi mail trả lời cho người dùng.\\
Tham số: 
\begin{itemize}
\item \lstinline{ip_address}: kiểu tuple(str, int) chứa địa chỉ ip và port của client gửi tới.
\item \lstinline{time}: thời gian chụp màn hình (giây).
\end{itemize}
Không có giá trị trả về. 
\item Ở client:
\begin{lstlisting}
def capture_screen(client):
\end{lstlisting}
Chức năng: nhận yêu cầu chụp màn hình từ server, chụp màn hình liên tục rồi gửi từng ảnh lại cho server.\\
Tham số: 
\begin{itemize}
\item \lstinline{client}: socket kết nối server với client.
\end{itemize}
Không có giá trị trả về.\\
Thư viện sử dụng: ImageGrab, io, time.\\
Ngoài ra, ở server còn có hàm sau dùng để tạo video từ những ảnh chụp màn hình nhận được từ client, sử dụng thư viện os, cv2.
\begin{lstlisting}
def create_video(image_folder: str):
\end{lstlisting}

\end{itemize}


